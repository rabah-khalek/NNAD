%% LyX 2.0.3 created this file.  For more info, see http://www.lyx.org/.
%% Do not edit unless you really know what you are doing.
\documentclass[twoside,english]{paper}
\usepackage{lmodern}
\renewcommand{\ttdefault}{lmodern}
\usepackage[T1]{fontenc}
\usepackage[latin9]{inputenc}
\usepackage[a4paper]{geometry}
\geometry{verbose,tmargin=3cm,bmargin=2.5cm,lmargin=2cm,rmargin=2cm}
\usepackage{color}
\usepackage{babel}
\usepackage{float}
\usepackage{bm}
\usepackage{amsthm}
\usepackage{amsmath}
\usepackage{amssymb}
\usepackage{graphicx}
\usepackage{esint}
\usepackage[unicode=true,pdfusetitle,
 bookmarks=true,bookmarksnumbered=false,bookmarksopen=false,
 breaklinks=false,pdfborder={0 0 0},backref=false,colorlinks=false]
 {hyperref}
\usepackage{breakurl}

\makeatletter

%%%%%%%%%%%%%%%%%%%%%%%%%%%%%% LyX specific LaTeX commands.
%% Because html converters don't know tabularnewline
\providecommand{\tabularnewline}{\\}

%%%%%%%%%%%%%%%%%%%%%%%%%%%%%% Textclass specific LaTeX commands.
\numberwithin{equation}{section}
\numberwithin{figure}{section}

%%%%%%%%%%%%%%%%%%%%%%%%%%%%%% User specified LaTeX commands.
\usepackage{babel}

\@ifundefined{showcaptionsetup}{}{%
 \PassOptionsToPackage{caption=false}{subfig}}
\usepackage{subfig}
\makeatother

\begin{document}

\title{Backward propagation with Mellin convolutions}

\maketitle

\section{Derivative of the $\chi^2$}

For the sake of simplicity we consider an uncorrelated $\chi^2$ for
one single experimental point with central value $F$ and standard
deviation $\sigma$:
\begin{equation}
\chi^2 = \left(\frac{\hat{F}-F}{\sigma}\right)^2\,,
\end{equation}
where $\hat{F}$ is the corresponding theoretical prediction. $\hat{F}$
is typically computed as a convolution integral between two distinct
sets of quantities $C_i$ and $N_i$:
\begin{equation}
\hat{F}\equiv \sum_i C_i \otimes N_i = \mathbf{C}\otimes\mathbf{N}\,,
\end{equation}
The details of the convolution sign $\otimes$ are not important, it
suffices to know that it involves an integral over a set of input
variables $\{\xi_p\}$. Therefore, the theoretical prediction $\hat{F}$
can be regarded as a \textit{functional} of the functions
$N_i$\footnote{In fact, $\hat{F}$ is also a functional of the
  functions $C_i$ but we assume these functions to be given.},
\textit{i.e.}  $\hat{F}\equiv \hat{F}[\{N_i\}]$. It follows that the
$\chi^2$ is also a functional of the functions $N_i$, \textit{i.e.}
$\chi^2\equiv \chi^2[\{N_i\}]$. In the case we are interested in, the
functions $N_i$ are the outputs of a feed-forward neural network with
$L$ layers (including input and output layers) parametrised by a set
of weights $\omega_{ij}^{(\ell)}$ and biases $\theta_{i}^{(\ell)}$. We
assume that the nodes of the $\ell$-th layer have all the same
activation function $\sigma_\ell$ associated. The output of the neural
network can be written as:
\begin{equation}
\begin{array}{rcl}
\displaystyle N_i\equiv
  N_i(\{\xi_p\};\{\omega_{ij}^{(\ell)},\theta_{i}^{(\ell)}\}) &=&\displaystyle
                                                          \sigma_L\left(\sum_{j^{(1)}}^{N_{L-1}}\omega_{i
                                                          j^{(1)}}^{(L)}y_{j^{(1)}}^{(L-1)}+\theta_{i}^{(L)}\right)\\
\\
&=&\displaystyle
                                                          \sigma_L\left(\sum_{j^{(1)}=1}^{N_{L-1}}\omega_{i
                                                          j^{(1)}}^{(L)}\sigma_{L-1}\left(\sum_{j^{(2)}=1}^{N_{L-2}}\omega_{j^{(1)}
                                                          j^{(2)}}^{(L)}y_{j^{(2)}}^{(L-2)}+\theta_{j^{(1)}}^{(L-1)}\right)+\theta_{i}^{(L)}\right)\\
&=&\dots
\end{array}
\end{equation}

We want to compute the following derivatives:
\begin{equation}
\frac{\partial \chi^2}{\partial \omega_{ij}^{(\ell)}} =
2\left(\frac{\mathbf{C}\otimes\mathbf{N}-F}{\sigma^2}\right)
\mathbf{C}\otimes \frac{\partial \mathbf{N}}{\partial
  \omega_{ij}^{(\ell)}}\,,
\label{eq:deromega}
\end{equation}
and:
\begin{equation}
\frac{\partial \chi^2}{\partial \theta_{i}^{(\ell)}} =
2\left(\frac{\mathbf{C}\otimes\mathbf{N}-F}{\sigma^2}\right)
\mathbf{C}\otimes \frac{\partial \mathbf{N}}{\partial
  \theta_{i}^{(\ell)}}\,.
\label{eq:dertheta}
\end{equation}
Let us first focus on Eq.~(\ref{eq:deromega}). We define:
\begin{equation}
\begin{array}{l}
  \displaystyle x_i^{(\ell)} =
  \sum_{j=1}^{N_{\ell-1}}\omega_{ij}^{(\ell)}y_{j}^{(\ell-1)}+\theta_{i}^{(\ell)}\,,\\
  \\
  \displaystyle y_i^{(\ell)} =
  \sigma_\ell\left(x_i^{(\ell)}\right)\,,\\
  \\
  \displaystyle z_i^{(\ell)} = \sigma'_\ell\left(x_i^{(\ell)}\right)\,,
\end{array}
\label{eq:definitions}
\end{equation}
so that we can apply the chain rule:
\begin{equation}
\begin{array}{rcl}
\displaystyle  \frac{\partial N_k}{\partial \omega_{ij}^{(\ell)}}&=&\displaystyle\frac{\partial
    y_k^{(L)}}{\partial \omega_{ij}^{(\ell)}}\\
\\
&=&\displaystyle z_k^{(L)}
  \frac{\partial x_k^{(L)}}{\partial \omega_{ij}^{(\ell)}}\\
\\
&=&\displaystyle  \sum_{j^{(1)}=1}^{N_{L-1}}  \left[z_k^{(L)}\omega_{k j^{(1)}}^{(L)}\right]\frac{\partial
    y_{j^{(1)}}^{(L-1)}}{\partial \omega_{ij}^{(\ell)}}\\
\\
&=&\displaystyle  \sum_{j^{(1)}=1}^{N_{L-1}} \sum_{j^{(2)}=1}^{N_{L-2}} \left[z_k^{(L)} \omega_{k j^{(1)}}^{(L)}
  \right]\left[z_{j^{(1)}}^{(L-1)}\omega_{j^{(1)} j^{(2)}}^{(L-1)}\right]\frac{\partial
    y_{j^{(2)}}^{(L-2)}}{\partial \omega_{ij}^{(\ell)}}\\
\\
&=&\displaystyle \dots
\end{array}
\end{equation}
As evident, the chain rule penetrates into the neural network starting
from the output layer all the way back until the $\ell$-th layer (that
is, the layer where the parameter $\omega_{ij}^{(\ell)}$ with respect
to which we are deriving belongs to). In order to write the formula we
are looking for in a closed form we define:
\begin{equation}
z_i^{(\ell)}\omega_{ij}^{(\ell)}=S_{ij}^{(\ell)} \left(= \frac{\partial y_i^{(\ell)}}{\partial
  y_j^{(\ell-1)}}\right) \,,
\end{equation}
and using the matricial form we can write:
\begin{equation}
\frac{\partial \mathbf{N}}{\partial \omega_{ij}^{(\ell)}} =
\mathbf{S}^{(L)}\cdot \mathbf{S}^{(L-1)}\cdots
\mathbf{S}^{(\ell+1)}\cdot\frac{\partial \mathbf{y}^{(\ell)}}{\partial
\omega_{ij}^{(\ell)}}\,,
\end{equation}
that can be written in a more compact form as:
\begin{equation}
\frac{\partial \mathbf{N}}{\partial \omega_{ij}^{(\ell)}} =
\left[\prod_{\alpha=L}^{\ell+1}\mathbf{S}^{(\alpha)}\right]\cdot\frac{\partial \mathbf{y}^{(\ell)}}{\partial
\omega_{ij}^{(\ell)}}\,.
\end{equation}
In addition, the derivative in the r.h.s. can be computed explicitly
and reads:
\begin{equation}
\frac{\partial y_k^{(\ell)}}{\partial
\omega_{ij}^{(\ell)}} = z_k^{(\ell)} \frac{\partial x_k^{(\ell)}}{\partial
\omega_{ij}^{(\ell)}} = \delta_{ki} z_i^{(\ell)}y_j^{(\ell-1)}\,.
\label{eq:dydom}
\end{equation}
It is simple to see that the derivative in Eq.~(\ref{eq:dertheta})
takes the form:
\begin{equation}
\frac{\partial \mathbf{N}}{\partial \theta_{i}^{(\ell)}} =
\left[\prod_{\alpha=L}^{\ell+1}\mathbf{S}^{(\alpha)}\right]\cdot\frac{\partial \mathbf{y}^{(\ell)}}{\partial
\theta_{i}^{(\ell)}}\,,
\end{equation}
with:
\begin{equation}
\frac{\partial y_k^{(\ell)}}{\partial
\theta_{i}^{(\ell)}} = \delta_{ki}z_i^{(\ell)}\,.
\label{eq:dydth}
\end{equation}
The presence of $\delta_{ki}$ in both Eqs.~(\ref{eq:dydom})
and~(\ref{eq:dydth}) simplifies the computation of the derivatives
yielding:
\begin{equation}
\frac{\partial N_k}{\partial \theta_{i}^{(\ell)}} =
{\Sigma}_{ki}^{(\ell)}
z_i^{(\ell)}\qquad\mbox{and}\qquad
\frac{\partial N_k}{\partial \omega_{ij}^{(\ell)}} ={\Sigma}_{ki}^{(\ell)}
z_i^{(\ell)} y_j^{(\ell-1)}\,.
\end{equation}
In both cases, the key quantities to be computes are the matrices:
\begin{equation}
\mathbf{\Sigma}^{(\ell)} = \prod_{\alpha=L}^{\ell+1}\mathbf{S}^{(\alpha)}\,,
\end{equation}
that can be computed recursively moving backward from the output layer
as:
\begin{equation}
\mathbf{\Sigma}^{(\ell-1)} = \mathbf{\Sigma}^{(\ell)} \cdot \mathbf{S}^{(\ell)}\,,
\end{equation}
starting from:
\begin{equation}
\mathbf{\Sigma}^{(L)} = \mathbf{I}\,.
\end{equation}

The same technology can be used to compute derivatives of the neural
network w.r.t. the input variables $\{\xi_p\}$. Indeed, a
straightforward application of the chain rule discussed above produces
the compact result:
\begin{equation}
\frac{\partial \mathbf{N}}{\partial {\bm \xi}} = \prod_{\alpha=L}^{1}\mathbf{S}^{(\alpha)}=\mathbf{\Sigma}^{(0)}\,,
\end{equation}
or, making the indices explicit:
\begin{equation}
\frac{\partial N_k}{\partial \xi_p} = \Sigma_{kp}^{(0)}\,.
\end{equation}

A generalisation of feed-forward neural network that might be useful
considering is one in which the linear combination of weights, biases,
and inputs form the preceding layer that enter the computation of a
given node (see first identity of Eq.~(\ref{eq:definitions})) is
replaced by:
\begin{equation}
  \displaystyle x_i^{(\ell)} =
  \sum_{j=1}^{N_{\ell-1}}f(\omega_{ij}^{(\ell)})y_{j}^{(\ell-1)}+\theta_{i}^{(\ell)}\,,
\end{equation}
where $f$ is some derivable function. This change has the effect of
changing the matrices $\mathbf{S}^{(\ell)}$ as follows:
\begin{equation}
S_{ij}^{(\ell)} = z_i^{(\ell)}f(\omega_{ij}^{(\ell)})\,,
\end{equation}
and the derivatives w.r.t. the weight $\omega_{ij}^{(\ell)}$ as
follows:
\begin{equation}
  \frac{\partial N_k}{\partial \omega_{ij}^{(\ell)}} ={\Sigma}_{ki}^{(\ell)}
  z_i^{(\ell)} y_j^{(\ell-1)}f'(\omega_{ij}^{(\ell)})\,.
\end{equation}


\end{document}
